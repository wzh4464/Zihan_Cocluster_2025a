% arara: pdflatex: { options: ["--synctex=1", "-interaction=nonstopmode"] }
% arara: bibtex
% arara: pdflatex: { options: ['-synctex=1', '-interaction=nonstopmode'] }
% arara: pdflatex: { options: ['-synctex=1', '-interaction=nonstopmode'] }
%%
%% Copyright 2022 OXFORD UNIVERSITY PRESS
%%
%% This file is part of the 'oup-authoring-template Bundle'.
%% ---------------------------------------------
%%
%% It may be distributed under the conditions of the LaTeX Project Public
%% License, either version 1.2 of this license or (at your option) any
%% later version.  The latest version of this license is in
%%    http://www.latex-project.org/lppl.txt
%% and version 1.2 or later is part of all distributions of LaTeX
%% version 1999/12/01 or later.
%%
%% The list of all files belonging to the 'oup-authoring-template Bundle' is
%% given in the file `manifest.txt'.
%%
%% Template article for OXFORD UNIVERSITY PRESS's document class `oup-authoring-template'
%% with bibliographic references
%%

%%%CONTEMPORARY%%%
%\documentclass[unnumsec,webpdf,contemporary,large]{oup-authoring-template}%
\documentclass[unnumsec,webpdf,contemporary,large,namedate]{oup-authoring-template}% uncomment this line for author year citations and comment the above
%\documentclass[unnumsec,webpdf,contemporary,medium]{oup-authoring-template}
%\documentclass[unnumsec,webpdf,contemporary,small]{oup-authoring-template}

%%%MODERN%%%
%\documentclass[unnumsec,webpdf,modern,large]{oup-authoring-template}
%\documentclass[unnumsec,webpdf,modern,large,namedate]{oup-authoring-template}% uncomment this line for author year citations and comment the above
%\documentclass[unnumsec,webpdf,modern,medium]{oup-authoring-template}
%\documentclass[unnumsec,webpdf,modern,small]{oup-authoring-template}

%%%TRADITIONAL%%%
%\documentclass[unnumsec,webpdf,traditional,large]{oup-authoring-template}
%\documentclass[unnumsec,webpdf,traditional,large,namedate]{oup-authoring-template}% uncomment this line for author year citations and comment the above
%\documentclass[unnumsec,namedate,webpdf,traditional,medium]{oup-authoring-template}
%\documentclass[namedate,webpdf,traditional,small]{oup-authoring-template}

%\onecolumn % for one column layouts

%\usepackage{showframe}

\graphicspath{{Fig/}}

%% my changes
\usepackage{hyperref} % for hyperlinks
\usepackage{cleveref} % for \cref

% line numbers
%\usepackage[mathlines, switch]{lineno}
%\usepackage[right]{lineno}

\theoremstyle{thmstyleone}%
\newtheorem{theorem}{Theorem}%  meant for continuous numbers
%%\newtheorem{theorem}{Theorem}[section]% meant for sectionwise numbers
%% optional argument [theorem] produces theorem numbering sequence instead of independent numbers for Proposition
\newtheorem{proposition}[theorem]{Proposition}%
%%\newtheorem{proposition}{Proposition}% to get separate numbers for theorem and proposition etc.
\theoremstyle{thmstyletwo}%
\newtheorem{example}{Example}%
\newtheorem{remark}{Remark}%
\theoremstyle{thmstylethree}%
\newtheorem{definition}{Definition}

\begin{document}

\journaltitle{Journal Title Here}
\DOI{DOI HERE}
\copyrightyear{2022}
\pubyear{2019}
\access{Advance Access Publication Date: Day Month Year}
\appnotes{Paper}

\firstpage{1}

%\subtitle{Subject Section}

\title[Tensor Co-clustering for C. elegans Morphogenesis]{Discovering Coordinated Cell Behavior in C. elegans Morphogenesis with Tensor Co-clustering}

\author[1,$\ast$]{Wu Zihan}
\author[1]{Zhaoke Huang}
\author[1]{Zelin Li}
\author[1]{Hong Yan}

\authormark{Wu et al.}

\address[1]{\orgdiv{Department of Electrical Engineering}, \orgname{City University of Hong Kong}, \orgaddress{\country{Hong Kong SAR, China}}}

\corresp[$\ast$]{Corresponding author. \href{mailto:wzh4464@gmail.com}{wzh4464@gmail.com}}

\received{Date}{0}{Year}
\revised{Date}{0}{Year}
\accepted{Date}{0}{Year}

%\editor{Associate Editor: Name}

\abstract{
\textbf{Motivation:} The embryonic development of the nematode Caenorhabditis elegans is a cornerstone of developmental biology, offering a complete and invariant cell lineage. Modern 4D microscopy captures this entire process at single-cell resolution, generating vast spatio-temporal datasets. A fundamental challenge is to automatically identify groups of cells that exhibit coordinated behaviors over specific time intervals, which are the functional signatures of complex morphogenetic events. Traditional analysis methods, which often rely on representing this multi-modal data in matrices, can obscure the higher-order relationships between cells, their features, and their dynamics over time.\\
\textbf{Results:} We introduce a novel computational framework, Tensor-AHPM, that extends the Adaptive Hierarchical Partitioning and Merging (AHPM) co-clustering algorithm from matrices to tensors. This allows for the direct analysis of multi-way spatio-temporal data (cells × time × features). Our method preserves the theoretical probabilistic guarantees of the original AHPM, ensuring robust identification of co-clusters. We applied our framework to a publicly available C. elegans embryonic dataset, using a rich feature set including cell shape, kinematics, and biological markers. Without prior biological knowledge, our method successfully and automatically identified two critical, well-documented morphogenetic events: the collective intercalation of dorsal epithelial cells and the clustering and polarization of the intestinal primordium cells.\\
\textbf{Conclusion:} Tensor-AHPM provides a powerful, data-driven approach to discover significant spatio-temporal patterns in complex biological systems. It overcomes the limitations of matrix-based tools and offers a robust framework for generating testable hypotheses from high-dimensional developmental data.\\
\textbf{Availability:} The software and analysis pipeline are available at [Link to GitHub Repository].}

\keywords{Tensor co-clustering, C. elegans, morphogenesis, spatio-temporal data, computational biology}

% \boxedtext{
% \begin{itemize}
% \item Key boxed text here.
% \item Key boxed text here.
% \item Key boxed text here.
% \end{itemize}}

\maketitle


\section{Introduction}

The process of morphogenesis, by which an organism develops its shape, is driven by a complex symphony of coordinated cellular behaviors, including migration, division, and changes in morphology and adhesion. The nematode Caenorhabditis elegans has long served as a pre-eminent model organism for studying these fundamental processes. Its largely invariant cell lineage, optical transparency, and rapid development allow for the complete tracking of every cell from fertilization to hatching using 4D (3D space + time) light microscopy~\citep{kaletta2006FindingFunctionNovel,hwang2003CaenorhabditisElegansEarly}.

The resulting datasets are a rich resource but present a formidable analytical challenge. They encapsulate the spatio-temporal trajectories and feature dynamics of hundreds of cells over thousands of time points. A key goal in analyzing this data is to move beyond tracking individual cells to identifying functional modules: groups of cells that act in a coordinated manner over specific periods of development. Such collective behaviors are the building blocks of tissue formation and organogenesis~\citep{setty2008FourdimensionalRealisticModeling}.

Traditional computational approaches often simplify this high-dimensional data by "flattening" it into a two-dimensional matrix, for instance, by concatenating time points or features. This process of matricization, however, fundamentally disrupts the inherent multi-way structure of the data. The intricate, higher-order interactions between cells, their dynamic features (e.g., velocity, shape), and time are obscured, limiting the discovery of complex patterns.

Tensors, or multi-dimensional arrays, provide a more natural and powerful framework for representing such data, preserving these relationships~\citep{sun2008IncrementalTensorAnalysis}. A tensor of dimensions (cells × time × features) can holistically capture the system's dynamics. To analyze this representation, we require computational tools capable of identifying meaningful sub-structures within the tensor.

In this work, we propose a novel framework for this purpose by extending the Adaptive Hierarchical Partitioning and Merging (AHPM) co-clustering algorithm from matrices to tensors. Co-clustering, or biclustering, simultaneously groups rows and columns of a matrix, and has proven highly effective in bioinformatics for finding local patterns, such as subsets of genes that are co-expressed under a subset of conditions~\citep{zapala2006MultivariateRegressionAnalysis}. Our tensor extension, Tensor-AHPM, identifies "co-clusters" that are cohesive blocks within the data tensor, corresponding to groups of cells exhibiting similar feature dynamics over contiguous time intervals. Critically, our extension maintains the strong probabilistic guarantees of the original AHPM for preserving co-cluster integrity during the analysis of large datasets.

To validate our approach, we applied it to a comprehensive 4D dataset of C. elegans embryogenesis. By analyzing a rich set of features including 3D shape descriptors, kinematics, and lineage information, our method successfully rediscovered two well-characterized morphogenetic events without any a priori biological annotation: dorsal intercalation and the formation of the intestinal primordium~\citep{bao2006AutomatedCellLineage,kaletta1997BinarySpecificationEmbryonic,artal-sanz2006CaenorhabditisElegansVersatile,sato2015LeftRightAsymmetric,stoeckius2009LargescaleSortingElegans}. This demonstrates that Tensor-AHPM is a powerful, unbiased tool for pattern discovery in high-dimensional developmental data, capable of generating concrete, biologically meaningful hypotheses.

\section{Background and Related Work}

Our work is situated at the intersection of three domains: the computational analysis of C. elegans morphogenesis, co-clustering algorithms, and tensor-based data mining in biology.

\subsection{Computational Analysis of C. elegans Morphogenesis}
The complete and stereotyped cell lineage of C. elegans has made it a focus for computational modeling. Early work focused on automated cell tracking and lineage tracing to create digital representations of the embryo~\citep{bao2006AutomatedCellLineage,murray2008AutomatedAnalysisEmbryonic}. Subsequent studies have leveraged this data to quantify various aspects of development. These analyses typically extract a wide array of quantitative features, including cell position, velocity, and acceleration; morphological descriptors like volume and sphericity; and relational features such as cell-cell contact areas~\citep{2023FastAccurateCell}.
% TODO: Zelin
These features serve as the basis for models of specific developmental events, but often require significant prior knowledge to select the relevant cells and time windows for analysis. Our work builds on this tradition of quantitative feature extraction but aims to discover the relevant cell groups and time periods automatically.

\subsection{Co-clustering for Biological Data Analysis}
Traditional clustering methods partition data along a single dimension (e.g., grouping cells based on their overall behavior). Co-clustering, also known as biclustering, offers a significant advantage by simultaneously clustering two dimensions~\citep{hochreiter2010FABIAFactorAnalysis,sill2011RobustBiclusteringSparse,orzechowski2018EBICEvolutionarybasedParallel,yi2021COBRACFastImplementation,kim2007SparseNonnegativeMatrix}. In genomics, this allows for the identification of subsets of genes that are co-regulated only under specific experimental conditions, a local pattern missed by global clustering. A multitude of co-clustering algorithms have been developed, each with different strengths~\citep{saelens2018ComprehensiveEvaluationModule}. The AHPM algorithm~\citep{wu2024ScalableCoclusteringLargescale} is a recent, highly scalable method designed for large matrices. Its key innovation is a probabilistic model that guides its partitioning strategy, providing a theoretical guarantee that true co-clusters will not be fragmented with a user-defined probability

\subsection{Tensor Decomposition in Bioinformatics}
To address the limitations of matrix-based analysis, tensor decompositions have emerged as a powerful tool in bioinformatics for analyzing multi-way data, such as patient data with multiple omics modalities (patients × genes × drugs) or longitudinal studies (patients × biomarkers × time)~\citep{kolda2009TensorDecompositionsApplications}. Methods like CANDECOMP/PARAFAC (CP) and Tucker decomposition are used to find latent patterns, reduce dimensionality, and impute missing values~\citep{yu2025OptimizationMethodsTensor}. Several studies have used these decompositions for clustering tasks, for instance, by first decomposing the tensor and then clustering the resulting factor vectors~\citep{cheng2019TensorBasedLowDimensionalRepresentation}. However, these approaches often have limitations: they typically require the number of clusters to be specified in advance, can be sensitive to noise, and do not inherently identify overlapping or hierarchically structured patterns common in biological systems.

By extending the probabilistic framework of AHPM to a tensor representation, our work directly addresses the limitations of both traditional matrix co-clustering and existing tensor factorization methods. It provides a scalable, robust, and data-driven approach specifically designed to discover significant spatio-temporal patterns of coordinated cell behavior in complex developmental systems.

\section{Conclusion}
Tensor-AHPM provides a powerful, data-driven approach to discover significant spatio-temporal patterns in complex biological systems. It overcomes the limitations of matrix-based tools and offers a robust framework for generating testable hypotheses from high-dimensional developmental data.

%%%%%%%%%%%%%%

\begin{appendices}

\section{Section title of first appendix}\label{sec11}

Nam dui ligula, fringilla a, euismod sodales, sollicitudin vel, wisi. Morbi auctor lorem non justo. Nam lacus libero,
pretium at, lobortis vitae, ultricies et, tellus. Donec aliquet, tortor sed accumsan bibendum, erat ligula aliquet magna,
vitae ornare odio metus a mi. Morbi ac orci et nisl hendrerit mollis. Suspendisse ut massa. Cras nec ante. Pellentesque
a nulla. Cum sociis natoque penatibus et magnis dis parturient montes, nascetur ridiculus mus. Aliquam tincidunt
urna. Nulla ullamcorper vestibulum turpis. Pellentesque cursus luctus mauris.

\end{appendices}

\section{Competing interests}
No competing interest is declared.

\section{Author contributions statement}

W.Z. and H.Y. conceived the experiment(s), W.Z. conducted the experiment(s), W.Z., Z.H., and Z.L. analysed the results. All authors wrote and reviewed the manuscript.

\section{Acknowledgments}
The authors thank the anonymous reviewers for their valuable suggestions. This work is supported in part by funds from the National Science Foundation (NSF: \# 1636933 and \# 1920920).


\bibliographystyle{abbrvnat}
\bibliography{refs}


%USE THE BELOW OPTIONS IN CASE YOU NEED AUTHOR YEAR FORMAT.
%\bibliographystyle{abbrvnat}
%\bibliography{reference}

\end{document}
