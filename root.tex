% arara: pdflatex: { options: ["--synctex=1", "-interaction=nonstopmode"] }
% arara: bibtex
% arara: pdflatex: { options: ['-synctex=1', '-interaction=nonstopmode'] }
% arara: pdflatex: { options: ['-synctex=1', '-interaction=nonstopmode'] }
%%
%% Copyright 2022 OXFORD UNIVERSITY PRESS
%%
%% This file is part of the 'oup-authoring-template Bundle'.
%% ---------------------------------------------
%%
%% It may be distributed under the conditions of the LaTeX Project Public
%% License, either version 1.2 of this license or (at your option) any
%% later version.  The latest version of this license is in
%%    http://www.latex-project.org/lppl.txt
%% and version 1.2 or later is part of all distributions of LaTeX
%% version 1999/12/01 or later.
%%
%% The list of all files belonging to the 'oup-authoring-template Bundle' is
%% given in the file `manifest.txt'.
%%
%% Template article for OXFORD UNIVERSITY PRESS's document class `oup-authoring-template'
%% with bibliographic references
%%

%%%CONTEMPORARY%%%
%\documentclass[unnumsec,webpdf,contemporary,large]{oup-authoring-template}%
%\documentclass[numsec,webpdf,modern,large]{oup-authoring-template}% uncomment this line for author year citations and comment the above
%\documentclass[unnumsec,webpdf,contemporary,medium]{oup-authoring-template}
%\documentclass[unnumsec,webpdf,contemporary,small]{oup-authoring-template}

%%%MODERN%%%
%\documentclass[unnumsec,webpdf,modern,large]{oup-authoring-template}
\documentclass[unnumsec,webpdf,modern,large,namedate]{oup-authoring-template}% uncomment this line for author year citations and comment the above
%\documentclass[unnumsec,webpdf,modern,medium]{oup-authoring-template}
%\documentclass[unnumsec,webpdf,modern,small]{oup-authoring-template}

%%%TRADITIONAL%%%
%\documentclass[unnumsec,webpdf,traditional,large]{oup-authoring-template}
%\documentclass[unnumsec,webpdf,traditional,large,namedate]{oup-authoring-template}% uncomment this line for author year citations and comment the above
%\documentclass[unnumsec,namedate,webpdf,traditional,medium]{oup-authoring-template}
%\documentclass[namedate,webpdf,traditional,small]{oup-authoring-template}

%\onecolumn % for one column layouts

%\usepackage{showframe}

\graphicspath{{Fig/}}

% --- packages you added ---
\usepackage{booktabs}
\usepackage{amsmath, amssymb}
\usepackage{subfig}
\usepackage{hyperref}
\usepackage{cleveref}

% --- theorem styles already defined in your file ---
\theoremstyle{thmstyleone}\newtheorem{theorem}{Theorem}
\newtheorem{proposition}[theorem]{Proposition}
\theoremstyle{thmstyletwo}\newtheorem{example}{Example}
\newtheorem{remark}{Remark}
\theoremstyle{thmstylethree}\newtheorem{definition}{Definition}

\begin{document}

\journaltitle{Bioinformatics}
\DOI{DOI HERE}
\copyrightyear{2025}
\pubyear{2025}
\access{Advance Access Publication Date: Day Month Year}
\appnotes{Original Paper}

\firstpage{1}

\title[DiMergeTCC: Theory-Guided Tensor Co-clustering for C. elegans]{DiMergeTCC: Theory-Guided Tensor Co-clustering Reveals Coordinated Cell Modules in \textit{C.~elegans} Morphogenesis}

\author[1,$\ast$]{Zihan Wu}
\author[1]{Zhaoke Huang}
\author[1]{Hong Yan}

\authormark{Wu et al.}

\address[1]{\orgdiv{Department of Electrical Engineering}, \orgname{City University of Hong Kong}, \orgaddress{\country{Hong Kong SAR, China}}}

\corresp[$\ast$]{Corresponding author. \href{mailto:wzh4464@gmail.com}{wzh4464@gmail.com}}

\received{Date}{0}{Year}
\revised{Date}{0}{Year}
\accepted{Date}{0}{Year}

\abstract{
\textbf{Motivation:} The embryonic development of the nematode \textit{Caenorhabditis elegans} is captured by modern 4D microscopy, yielding rich spatio-temporal datasets. Automatically discovering groups of cells that behave coordinately over particular time windows remains challenging when matrix-based representations disrupt higher-order relationships.\\
\textbf{Results:} We introduce DiMergeTCC, a tensor co-clustering framework that extends Distributed Merge Co-clustering from matrices to three-mode tensors (cells $\times$ time $\times$ features). The method combines randomized candidate generation with calibrated preservation bounds under stated assumptions, parallel local three-way clustering, hierarchical merging via three-mode Jaccard similarity with per-mode thresholds, and FDR-controlled statistical testing (Benjamini--Yekutieli). We provide conservative, dependence-aware upper bounds on the miss probability for reconstructing true tri-blocks after $T_p$ candidate generations, together with an estimation-and-calibration protocol for principled parameter selection. Applied to \textit{C.~elegans} embryogenesis without event labels, DiMergeTCC automatically recovered dorsal intercalation and intestinal primordium organization. Quantitative validation shows 22--34\% improvement in literature time window alignment over tensor/matrix baselines at matched FDR, with velocity and shape attributions remaining stable under 10--20\% additional missingness.\\
\textbf{Availability:} Code and reproducible workflows are available at an anonymized repository (link in Supplementary; to be made public upon publication). Processed data and analysis artifacts are included; raw imaging data are available via the CMap consortium under their data usage agreement.\\
\textbf{Contact:} \href{mailto:wzh4464@gmail.com}{wzh4464@gmail.com}
}

\keywords{Tensor co-clustering, \textit{C. elegans}, morphogenesis, spatio-temporal data, computational biology}

\maketitle

% ------------------------ INTRODUCTION ------------------------
\section{Introduction}

The process of morphogenesis, by which an organism develops its shape, is driven by a complex symphony of coordinated cellular behaviors, including migration, division, and changes in morphology and adhesion. The nematode Caenorhabditis elegans has long served as a pre-eminent model organism for studying these fundamental processes. Its largely invariant cell lineage, optical transparency, and rapid development allow for the complete tracking of every cell from fertilization to hatching using 4D (3D space + time) light microscopy~\citep{kaletta2006FindingFunctionNovel,hwang2003CaenorhabditisElegansEarly}.

The resulting datasets are a rich resource but present a formidable analytical challenge. They encapsulate the spatio-temporal trajectories and feature dynamics of hundreds of cells over thousands of time points. A key goal in analyzing this data is to move beyond tracking individual cells to identifying functional modules: groups of cells that act in a coordinated manner over specific periods of development. Such collective behaviors are the building blocks of tissue formation and organogenesis~\citep{setty2008FourdimensionalRealisticModeling,carvalho2020game}.

Traditional computational approaches often simplify this high-dimensional data by "flattening" it into a two-dimensional matrix, for instance, by concatenating time points or features. This process of matricization, however, fundamentally disrupts the inherent multi-way structure of the data \citep{kolda2009TensorDecompositionsApplications}. The intricate, higher-order interactions between cells, their dynamic features (e.g., velocity, shape), and time are obscured, limiting the discovery of complex patterns \citep{cichocki2015TensorDecompositionsSignal}.

Tensors, or multi-dimensional arrays, provide a more natural and powerful framework for representing such data, preserving these relationships~\citep{sun2008IncrementalTensorAnalysis,kolda2009TensorDecompositionsApplications}. A tensor of dimensions (cells $\times$ time $\times$ features) can holistically capture the system's dynamics. To analyze this representation, we require computational tools capable of identifying meaningful sub-structures within the tensor \citep{sidiropoulos2017TensorDecompositionSignal}.

In this work, we propose a novel framework for this purpose by extending the Distributed Merge Co-clustering (DiMergeCo) algorithm from matrices to tensors~\citep{wu2024scalable}. Co-clustering, or biclustering, simultaneously groups rows and columns of a matrix, and has proven highly effective in bioinformatics for finding local patterns, such as subsets of genes that are co-expressed under a subset of conditions~\citep{hartigan1972DirectClusteringData,madeira2004BiclusteringAlgorithmsBiological}. Our tensor extension, DiMergeTCC, identifies "co-clusters" that are cohesive blocks within the data tensor, corresponding to groups of cells exhibiting similar feature dynamics over contiguous time intervals. Critically, our extension maintains a calibrated preservation analysis (with explicit assumptions and estimation procedures) analogous in spirit to DiMergeCo, aimed at preserving co-cluster integrity during the analysis of large datasets.

\textbf{Contributions.} The main contributions of this work are:
\begin{enumerate}
\item \textbf{Methodological:} We propose DiMergeTCC, extending distributed merge co-clustering from matrices to cell $\times$ time $\times$ feature three-mode tensors. The method comprises randomized candidate generation, parallel local three-way clustering, and hierarchical merging based on three-mode Jaccard similarity with per-mode thresholds and non-decreasing composite score with uncertainty control. We provide conservative upper bounds for miss probability of complete preservation and merging-based reconstruction of true blocks across $T_p$ candidates, with explicit dependence correction and an estimation-and-calibration protocol, offering actionable guidance (not absolute guarantees) for large-scale unsupervised discovery.
\item \textbf{Statistical assessment:} We introduce empirical null distributions based on circular time permutation and feature shuffling, augmented with per-cell independent shifts, phase randomization, and block bootstrap to respect dependence, employing the Benjamini--Yekutieli procedure for FDR control. Robustness is systematically evaluated through leave-one-embryo-out validation, consistency Jaccard measures, and missingness sensitivity analysis.
\item \textbf{Biological validation:} Without event labels, we automatically recover the temporal windows, velocity fields, and morphological attributions for dorsal intercalation and intestinal primordium organization. Compared to tensor/matrix baselines, time window alignment achieves 22--34\% improvement at matched FDR, with attributions remaining stable under 10--20\% additional missingness.
\end{enumerate}

\section{Methods}\label{sec:methods}

\subsection{Overview}
We present DiMergeTCC (Distributed Merge Tensor Co-Clustering), a probabilistically-guaranteed tensor co-clustering method that extends DiMergeCo from matrices to three-mode tensors. The algorithm comprises four main stages:

\begin{enumerate}
\item \textbf{Randomized candidate generation:} Generate $T_p$ independent tri-mode partitions (cell subsets $\times$ time windows $\times$ feature groups) using lineage/spatial proximity weighting, multi-scale temporal windows (16/32/64 frames), and dependency-aware feature grouping.

\item \textbf{Parallel local three-way clustering:} For each candidate partition, perform mode-3 unfolding followed by spectral co-clustering (SCC) or projective nonnegative matrix tri-factorization (PNMTF) to identify locally coherent cell--time biclusters, then regroup features by within-block variance contribution.

\item \textbf{Hierarchical merging and deduplication:} Insert all local tri-blocks into a priority queue ranked by composite score (fitting error + regularization). Merge blocks with high three-mode Jaccard overlap only if per-mode overlaps exceed thresholds and the merger maintains or increases the composite score within a non-decreasing confidence interval.

\item \textbf{Statistical testing and FDR control:} Apply circular time shifts and feature shuffling to construct empirical null distributions for alignment scores, velocity biases, and feature attributions (augmented nulls described below). Control false discovery rate using the Benjamini--Yekutieli procedure across all discovered tri-clusters.
\end{enumerate}

Each stage is designed to preserve the probabilistic guarantees: true tri-blocks are likely to be sampled intact (Stage 1), identified locally (Stage 2), merged correctly (Stage 3), and validated rigorously (Stage 4).

\subsection{Data acquisition and imaging/segmentation pipeline}
We used the CMap platform to obtain \textit{C.~elegans} embryonic 4D imaging and cell-level morphological data. Embryos co-expressed a nuclear marker (HIS-72::GFP) and a membrane marker (mCherry-PH(PLC1$\delta$1)) and were imaged on a Leica SP8 confocal microscope. Each session acquired two channels across \textbf{92 $z$-planes}, with $x/y$ resolution of 0.09~$\mu$m/pixel, $z$-resolution of 0.42~$\mu$m/pixel, and a temporal resolution of $\sim$1.5~min/frame. The full pipeline included deconvolution, automated lineage tracking (StarryNite/AceTree), and membrane segmentation (EDT-DMFNet), producing for each time point: \textbf{cell identity, lineage/fate, shape descriptors, volume ($V$), surface area ($A_S$), and cell--cell contact area ($A_C$)}.

To improve data quality, the raw pipeline incorporates strategies for handling abnormal voxel clusters (e.g., apoptotic ``solid'' fluorescence blobs) and membrane undersampling due to low $z$-resolution. Acquisition parameters were tuned to balance signal intensity and phototoxicity (block scanning, pinhole and laser power compensation). The resulting morphological maps cover $>95\%$ of cells with $<8\%$ missingness.

\subsection{Feature extraction}

\subsubsection{Basic morphological quantities}
From each 3D segmented cell, we compute:
\begin{itemize}
    \item Volume $V$: voxel count $\times$ voxel size.
    \item Surface area $A_S$: alpha-shape triangulation of the membrane surface.
    \item Contact area $A_C$: total area of membrane--membrane interfaces with neighbors.
\end{itemize}
We adopt the \textbf{dimensionless sphericity (isoperimetric quotient)}; in ablations replacing sphericity with alternative shape metrics (e.g., elongation ratio, mean curvature) we observe similar qualitative trends with modest quantitative differences (Supplementary Table~S3).
\begin{equation}
\Psi \;=\; \frac{\pi^{1/3}\,(6V)^{2/3}}{A_S}
\label{eq:sphericity}
\end{equation}
which equals 1 for a perfect sphere and decreases with deformation; it serves as a physically meaningful, unitless proxy for shape irregularity during migration/intercalation.

\subsubsection{3DCSQ shape spectral features}
To obtain compact and comparable shape descriptors, we follow the \textbf{3DCSQ} approach: each membrane surface is parameterized to the sphere and expanded in spherical harmonics, followed by PCA to extract three types of vectors:
\begin{enumerate}
    \item \textit{eigengrid}$_k$: principal components of the spherical mesh weights.
    \item \textit{eigenharmonic}$_k$: principal components of spherical harmonic coefficients.
    \item \textit{eigenspectrum}$_\ell$: power spectrum coefficients of the harmonics.
\end{enumerate}
These features preserve local detail while enabling reconstruction; we distinguish \textbf{static} features $f_{\text{static}}$ (per-frame) and \textbf{lifetime-aggregated} features $f_{\text{agg}}$ (per-cell summaries over the full tracked lifetime).  In addition, to capture temporal dynamics explicitly, we compute for selected features per-cell \emph{linear trend (slope)}, \emph{total variation}, and \emph{band-limited power} (via periodogram with 0.01--0.1 Hz equivalent bands at 1.5~min sampling); these dynamic summaries are standardized and included in the feature tensor.

\subsubsection{Spatial coordinates and kinematics}
For each cell at each time point, we compute 3D coordinates $(x,y,z)$ in the embryo frame; instantaneous velocity $\mathbf v=(v_x,v_y,v_z)$ and acceleration $\mathbf a=(a_x,a_y,a_z)$ via first- and second-order finite differences at 1.5~min resolution; speed $\|\mathbf v\|$ and acceleration magnitude $\|\mathbf a\|$; and motion direction represented by the unit vector $\hat{\mathbf v}$ and by azimuth $\phi$ and elevation $\psi$. All quantities are standardized and included in the feature tensor.

\subsubsection{Temporal and lineage alignment}
We align each embryo's timeline so that the end of the 4-cell stage is $t=0$ (corresponding to approximately 150-160 minutes post-fertilization). All time references throughout this paper are relative to this $t=0$ reference point unless explicitly noted otherwise. All datasets are resampled to a common frame rate (1.5~min) by linear interpolation; lineage mappings follow CMap's identity files. Missing frames are filled by nearest-neighbor interpolation and masked to avoid artificial correlations in co-clustering. To check against potential bias from canonical-axis alignment, we repeat all analyses with random small rigid jitters (rotation/translation) during preprocessing; results remain within reported confidence intervals (see Supplementary Fig.~S1b).

\subsubsection{Coordinate system convention}
We adopt a standardized embryonic coordinate system based on intrinsic biological axes: $x$ = anterior--posterior (AP), $y$ = left--right (LR), and $z$ = dorsal--ventral (DV). For each embryo, we align coordinates to this canonical frame using landmark-based rigid alignment (Procrustes): AP is estimated from anterior/posterior landmark centroids, LR from left/right hypodermal cohorts, and DV from the dorsal hypodermal centroid relative to the embryo centroid; velocities and positions are then transformed into this canonical frame. Negative $v_z$ denotes ventral/inward movement along the DV axis in this standardized frame. A schematic is provided in Supplementary Fig.~S1. All kinematic features are computed with first- and second-order finite differences at a 1.5~min sampling interval in this canonical coordinate system.

\subsection{Feature dictionary}
Table~\ref{tab:feature-dict} lists the complete feature set used for tensor construction. Morphological derivatives ($\Delta$) are computed as frame-to-frame differences. Lifetime-aggregated features are per-cell summaries computed over the full tracked lifetime (e.g., median/variance across time) and used for interpretation rather than clustering unless stated.

\begin{table*}[t]
\centering
\caption{Feature dictionary for tensor construction.}
\label{tab:feature-dict}
\begin{tabular}{llp{8cm}}
\toprule
\textbf{Category} & \textbf{Name} & \textbf{Description} \\
\midrule
Morphology & $V$ & Volume (voxel count $\times$ voxel size) \\
 & $A_S$ & Surface area from alpha-shape triangulation \\
 & $A_C$ & Total membrane-membrane contact area with neighbors \\
 & $\Psi$ & Sphericity (isoperimetric quotient): $\pi^{1/3}(6V)^{2/3}/A_S$ \\
 & $\Delta V$, $\Delta A_S$, $\Delta A_C$ & Temporal derivatives of $V$, $A_S$, $A_C$ \\
\midrule
Kinematics & $x,y,z$ & 3D coordinates in embryo frame (AP, LR, DV) \\
 & $v_x,v_y,v_z$ & Instantaneous velocity components (finite differences, 1.5~min) \\
 & $a_x,a_y,a_z$ & Instantaneous acceleration components (second-order differences) \\
 & $\|\mathbf v\|,\|\mathbf a\|$ & Speed and acceleration magnitudes \\
 & $\hat{\mathbf v},\phi,\psi$ & Unit direction vector and spherical angles (azimuth, elevation) \\
\midrule
3DCSQ (static) & eigengrid$_{1\ldots K_g}$ & PC scores from spherical mesh vertex positions \\
 & eigenharmonic$_{1\ldots K_h}$ & PC scores from spherical harmonic coefficients \\
 & eigenspectrum$_{0\ldots L_s}$ & Power spectrum coefficients of spherical harmonics \\
\midrule
3DCSQ (lifetime-aggregated) & agg\_eigengrid$_{1\ldots K_g}$ & Lifetime-aggregated eigengrid components (per-cell summaries) \\
 & agg\_eigenharmonic$_{1\ldots K_h}$ & Lifetime-aggregated eigenharmonic components \\
 & agg\_eigenspectrum$_{0\ldots L_s}$ & Lifetime-aggregated eigenspectrum coefficients \\
\bottomrule
\end{tabular}
\end{table*}

In practice, we set $K_g = 15$, $K_h = 15$, and $L_s = 20$ based on variance explained ($\geq 95\%$) and reconstruction error $< 5\%$. We also report sensitivity to these truncation levels (Supplementary Fig.~S2c) and demonstrate robustness under random feature subsampling (30--80\% of features retained; Supplementary Table~S3).
 
\subsection{Tensor construction}
For each cell $c = 1,\dots,C$ and time $t = 1,\dots,T$, we concatenate all standardized features (z-score using discovery set parameters, frozen on validation embryos) to form $\mathbf{z}_{c,t} \in \mathbb{R}^F$. Here, the discovery and validation sets are disjoint groups of embryos; the discovery set is used only to fit unsupervised preprocessing parameters and tune hyperparameters, while validation embryos are held out for all statistical testing. Stacking over cells and time yields a third-order tensor:
\begin{equation}
\mathcal{X} \in \mathbb{R}^{C \times T \times F}, \quad \mathcal{X}(c,t,f) = z_{c,t}^{(f)}.
\label{eq:tensor-def}
\end{equation}
The feature set includes morphological, kinematic, 3DCSQ static, lifetime-aggregated, and dynamic summary features described above.
 
\subsection{Tensor Co-Clustering (TCC): Extending DiMergeCo to three modes}
 
\subsubsection{Objective and model}
We extend the \textbf{Distributed Merge Co-clustering} (DiMergeCo) framework from matrices to tensors to jointly discover \textbf{cell subsets $\times$ time windows $\times$ feature subsets}. We adopt a tensor block model: within each block $(\mathcal{I}_u,\mathcal{J}_u,\mathcal{K}_u)$, the entries are well-approximated by a low-rank or block-constant structure:
\begin{equation}
\min_{\{\mathcal{I}_u,\mathcal{J}_u,\mathcal{K}_u\}} \sum_{u=1}^{U} \left\| \mathcal{X}_{\mathcal{I}_u,\mathcal{J}_u,\mathcal{K}_u} - \hat{\mathcal{X}}_u \right\|_F^2
\label{eq:block-objective}
\end{equation}
subject to $\hat{\mathcal{X}}_u$ being low-rank/block-constant.

\subsubsection{Probabilistic partitioning in three modes}
Following DiMergeCo's "probability of preservation" principle, we stochastically generate:
\begin{itemize}
    \item \textbf{Cell-mode candidates}: weighted by lineage/space proximity and activity ($\Psi$, derivatives), to avoid splitting correlated cells.
    \item \textbf{Time-mode candidates}: multi-scale sliding windows ($16$, $32$, $64$ frames) to capture both short cell-cycle changes and longer organogenesis events.
    \item \textbf{Feature-mode candidates}: initialized by dependence measures (HSIC or distance correlation) thresholded via a held-out split or fixed, pre-specified thresholds; collinear components are sparsified.
\end{itemize}
By repeating the partitioning $T_p$ times, true blocks of size above a minimum threshold are likely ($\geq 1-\delta$) to appear intact in at least one candidate, reducing fragmentation risk.

\subsubsection{Local three-way clustering (parallel)}
For each candidate $(\mathcal{I},\mathcal{J},\mathcal{K})$:
\begin{enumerate}
    \item \textbf{Mode-3 unfolding + biclustering}: unfold along features to a $(\mathcal{I} \times \mathcal{J}) \times \mathcal{K}$ matrix and perform simultaneous cell--time biclustering (e.g., SCC or PNMTF).
    \item \textbf{Feature regrouping}: given cell--time sub-blocks, cluster features by within-block variance contribution or loading magnitude.
    \item \textbf{Low-rank estimation}: apply rank-1 or small-rank Tucker approximation to score each block (density, lift, variance explained).
\end{enumerate}
This stage is fully parallelizable; each worker processes distinct blocks.

\subsubsection{Hierarchical merging and deduplication}
All local blocks are inserted into a priority queue sorted by composite score. Blocks with high \textbf{three-mode Jaccard} overlap and non-decreasing score upon merging are merged until convergence, subject to per-mode minimum overlaps and uncertainty-aware score checks.

The \textbf{three-mode Jaccard similarity} between two tri-blocks $B_1 = (\mathcal{I}_1, \mathcal{J}_1, \mathcal{K}_1)$ and $B_2 = (\mathcal{I}_2, \mathcal{J}_2, \mathcal{K}_2)$ is defined as:
\begin{equation}
J_3(B_1,B_2) = \frac{|\mathcal{I}_1\cap \mathcal{I}_2|}{|\mathcal{I}_1\cup \mathcal{I}_2|} \cdot \frac{|\mathcal{J}_1\cap \mathcal{J}_2|}{|\mathcal{J}_1\cup \mathcal{J}_2|} \cdot \frac{|\mathcal{K}_1\cap \mathcal{K}_2|}{|\mathcal{K}_1\cup \mathcal{K}_2|}
\label{eq:trimode_jaccard}
\end{equation}
We additionally require per-mode overlaps $J_I,J_J,J_K$ to exceed thresholds $\alpha_I,\alpha_J,\alpha_K$ to avoid chain-merging when each mode only weakly overlaps; non-contiguous temporal patterns are handled by allowing unions of a small number of contiguous windows with an additive TV penalty (reported in Supplementary Methods), and we report their frequency and impact in Supplementary Table~S4.

The thresholds $\alpha_I,\alpha_J,\alpha_K$ are set to balance sensitivity and specificity of merging. In practice, we select them by cross-validation on discovery embryos using strictly unsupervised criteria (block stability and reconstruction error), never using biological labels or validation endpoints; we also impose a theory-informed lower bound $\alpha_m\geq 0.5$ per mode $m\in\{I,J,K\}$ to preclude chain-merging via small overlaps. Typical values are $\alpha_I\in[0.6,0.8]$, $\alpha_J\in[0.7,0.9]$ (time contiguity), and $\alpha_K\in[0.5,0.7]$ (feature dependence). Sensitivity analyses show results are stable within these ranges.

\paragraph{Composite score and merge acceptance criterion.} Let the blockwise loss be $\mathcal{L}$ in Eq.~\eqref{eq:objective}. We define a composite score $S_u := -\,\mathcal{L}(\hat{\mathcal{X}}_u)$ (Eq.~\ref{sec:objective}) so that higher is better. For a proposed merge $B_{\mathrm{merge}} = B_a \cup B_b$, we compute $\Delta S = S(B_{\mathrm{merge}}) - \max\{S(B_a), S(B_b)\}$. To quantify uncertainty, we perform $N=500$ block bootstraps along the cell dimension within $\mathcal{I}_a\cup\mathcal{I}_b$ (with time contiguity preserved) to obtain a bootstrap distribution for $\Delta S$. A merge is \emph{accepted} iff (i) the 5th percentile of the bootstrap distribution of $\Delta S$ is $\ge 0$, and (ii) $J_I\ge\alpha_I$, $J_J\ge\alpha_J$, $J_K\ge\alpha_K$. This rule ensures monotone improvement under uncertainty while preventing over-merging.

\subsection{Theoretical analysis and calibration}

\begin{proposition}[Calibrated tri-mode preservation bound]
\label{thm:trimode_preservation}
Let $B^\star=(\mathcal I^\star,\mathcal J^\star,\mathcal K^\star)$ be a contiguous tri-block satisfying the following assumptions:
\begin{itemize}
    \item[(A1)] \textbf{Cell-mode coherence}: cells in $\mathcal{I}^\star$ have pairwise correlation $\geq \rho_I$ and spatial/lineage proximity
    \item[(A2)] \textbf{Time-mode contiguity}: $\mathcal{J}^\star$ forms a contiguous interval with temporal autocorrelation $\geq \rho_J$
    \item[(A3)] \textbf{Feature-mode dependence}: features in $\mathcal{K}^\star$ surpass a dependence threshold (HSIC/dCor) within the tri-block
\end{itemize}

In one randomized candidate generation, define the (clipped) single-mode preservation terms
\begin{align}
\tilde q_I &:= \min\!\Big\{1,\; \Phi\!\Big(\tfrac{\sqrt{|\mathcal{I}^\star|} \, \rho_I}{\sigma_I}\Big)\Big\} \\
\tilde q_J &:= \min\!\Big\{1,\; 1 - \exp\!\Big(-\tfrac{|\mathcal{J}^\star|^2 \, \rho_J}{2\sigma_J^2}\Big)\Big\} \\
\tilde q_K &:= \underline{\pi}_K\,\in[0,1]
\end{align}
where $\underline{\pi}_K$ is a one-sided lower confidence bound (e.g., Jeffreys/Wilson) on the probability that a feature from $\mathcal{K}^\star$ is grouped with other $\mathcal{K}^\star$ features during \emph{feature-mode candidate generation} under the fixed dependence thresholding scheme. To estimate $\underline{\pi}_K$, we compute the empirical proportion of $\mathcal{K}^\star$ features that co-occur in the same feature group across candidate generations and then apply a one-sided lower confidence bound (Jeffreys or Wilson) to account for sampling uncertainty; this yields a valid lower bound even when the number of generations is moderate. Let $\kappa\in(0,1]$ be a dependence discount capturing cross-mode dependence beyond candidate-generation independence.

Then the event that $B^\star$ is unsplit and enqueued with score above $\tau$ occurs with probability at least
\begin{equation}
q \;\ge\; \max\Big\{\, \kappa\, \tilde q_I\, \tilde q_J\, \tilde q_K,\; \min(\tilde q_I,\tilde q_J,\tilde q_K) - \varepsilon_{\mathrm{dep}},\; 0\,\Big\},
\label{eq:q_lower}
\end{equation}
The choice between the multiplicative discount and the minimum-minus-penalty form depends on the estimated cross-mode dependence. When candidate generation across modes is approximately independent, we prefer the multiplicative form with $\kappa\approx 1$. Under stronger dependence, the minimum-minus-penalty form provides a more conservative bound.

In practice, we estimate $\varepsilon_{\mathrm{dep}}$ by measuring the empirical joint failure rate of the candidate-generation process via block bootstrap across embryos and features and subtracting the product of marginal failure rates; when samples are limited, we use an upper bound derived from mode-wise dependence statistics (e.g., HSIC/dCor between candidate indicators). A conservative choice is to default to the minimum-minus-penalty form when uncertainty is high.

After $T_p$ i.i.d. generations and monotone merging with per-mode thresholds $(\alpha_I,\alpha_J,\alpha_K)$ and the acceptance rule above, the probability that there exists a block $\hat B$ with $J_3(\hat B,B^\star)\ge \alpha$ and $J_I\ge\alpha_I, J_J\ge\alpha_J, J_K\ge\alpha_K$ is at least $1-(1-q)^{T_p}$. Thus, under the stated assumptions and the calibration procedure described below, choosing $T_p\ge \lceil \log\delta/\log(1-q)\rceil$ yields an upper bound on the miss probability $\le \delta$.
\end{proposition}

\begin{proof}[Proof sketch]
(i) Lower-bound $\tilde q_I,\tilde q_J$ via concentration inequalities for intra-block affinities and contiguous coverage; clip to $[0,1]$ for well-defined probabilities. Define $\tilde q_K$ as a validated lower bound on within-support grouping probability under fixed dependence thresholding in feature-mode candidate generation. (ii) Independence applies only to \emph{candidate-generation} in each mode; downstream clustering/merging may induce dependence, captured by the multiplicative discount $\kappa$ or the Bonferroni-type conservative term $\min(\cdot)-\varepsilon_{\mathrm{dep}}$. (iii) Monotonicity of the composite score upon union merges is enforced algorithmically with per-mode thresholds and the bootstrap-based acceptance rule, preventing adverse merges that could discard true blocks. (iv) Apply the union bound over $T_p$ independent trials to obtain the preservation guarantee for at least one candidate; monotone merging preserves or improves coverage under the stated constraints.
\end{proof}

\textbf{Assumptions and parameter definitions:} To make the bounds operationable, we define the noise scales and thresholds:
\begin{itemize}
\item $\sigma_I, \sigma_J$ represent the empirical standard deviations of cell correlation estimates and temporal autocorrelation estimates, respectively.
\item $\rho_J$ is defined as the lag-1 autocorrelation averaged within the time block $\mathcal{J}^\star$.
\item $\rho_I$ is the lower bound of pairwise correlations between cells within $\mathcal{I}^\star$.
\item $\underline{\pi}_K$ is obtained from discovery data by resampling candidate generations under fixed dependence thresholds; we report the one-sided lower confidence bound.
\end{itemize}

Candidate generation operates independently across the three modes; the overall bound uses a dependence discount $\kappa$ or a conservative Bonferroni-style term to account for residual cross-mode dependence.

\textbf{Practical guidance:} In the discovery set, we estimate $\hat q$ and a conservative lower bound $\underline q$ by plugging in $\kappa=1-\hat\varepsilon_{\mathrm{dep}}$ with $\hat\varepsilon_{\mathrm{dep}}$ obtained via block bootstrap across embryos and features, and using Jeffreys/Wilson lower bounds for $\tilde q_\bullet$. We set
\begin{equation}
T_p = \Big\lceil \frac{\log \delta}{\log(1-\underline q)} \Big\rceil, \quad \delta\in(0,1)
\label{eq:Tp-selection}
\end{equation}
optionally applying a safety factor $\underline q \leftarrow 0.8\,\underline q$ when sample sizes are small. We empirically calibrate the $\delta$--$T_p$ curve on semi-synthetic planted blocks and report the realized miss rate vs. $1-(1-\underline q)^{T_p}$ in Supplementary Fig.~S2, confirming the theory as an upper bound rather than a point predictor.

\subsection{Objective function and constraints}\label{sec:objective}

The complete optimization objective combines fitting error with regularization:
\begin{equation}
\mathcal{L}(\hat{\mathcal{X}}_u) = \|\mathcal{X}_{\mathcal{I}_u,\mathcal{J}_u,\mathcal{K}_u}-\hat{\mathcal{X}}_u\|_F^2 + \lambda_r \, \text{rank}_{\text{Tucker}}(\hat{\mathcal{X}}_u) + \lambda_c \, \text{TV}_t(\hat{\mathcal{X}}_u),
\label{eq:objective}
\end{equation}
where $\text{TV}_t$ is the $\ell_1$ total variation along the time mode using forward finite differences (isotropic in features/cells within each block unless stated), and $\lambda_r, \lambda_c$ are regularization parameters chosen via cross-validation.

\subsection{Computational complexity}

The algorithm has the following complexity bounds:
\begin{itemize}
\item \textbf{Candidate generation}: $O(T_p(C\log C + T\log T + F\log F))$ 
\item \textbf{Local clustering}: $O(\sum_u |\mathcal{I}_u||\mathcal{J}_u||\mathcal{K}_u| \cdot r)$ where $r$ is the target rank
\item \textbf{Hierarchical merging}: $O(M\log M)$ where $M$ is the number of candidate blocks, typically $M = \Theta(T_p\, \bar m)$ with $\bar m$ the mean number of local blocks per candidate
\end{itemize}

On a dual-socket 16-core CPU (2.9 GHz) with 64GB RAM, our end-to-end pipeline (candidate generation, local tri-clustering, merging, and statistics) runs in 2.6 $\pm$ 0.3 hours for $C=320$, $T=104$, $F=65$, $T_p=1000$ (5 repeated runs). Per-stage medians: candidate generation 18 min, local clustering 76 min, merging 24 min, statistics 33 min. Scaling with $T_p$ is near-linear up to $T_p=2000$ (Supplementary Fig.~S4).

\subsection{Statistical assessment and robustness}
\begin{itemize}
    \item \textbf{Permutation/shift tests}: default nulls use circular time shifts within embryos to preserve marginal autocorrelation; when strong autocorrelation or seasonal structure is detected, we switch to Fourier phase randomization; for heterogeneous-feature blocks we additionally apply per-cell independent shifts and block bootstrap across features. The null variant is pre-specified per analysis stage and reported alongside each $q$-value. We control FDR by Benjamini--Yekutieli across tests.
    \item \textbf{Individual consistency}: leave-one-embryo-out validation; report weighted Jaccard similarity across embryos.
    \item \textbf{Missing-data sensitivity}: inject 5--20\% extra missingness on top of the real mask and measure recovery rate and score variation.
\end{itemize}

\subsection{Implementation and reproducibility}
Morphological/contact quantities and $\Psi$ follow CMap's definitions; 3DCSQ features use the authors' public implementation (SPHARM+PCA). Our tensor is built primarily from per-frame static features; lifetime-aggregated and dynamic summary features are used in post hoc interpretation. Feature-mode dependence is assessed by HSIC/dCor with fixed thresholds or hold-out calibration; BY FDR control is reserved for post-hoc statistical testing of discovered blocks. All code is in Python/Rust with MPI-based parallelization; seeds are fixed. Preprocessing scripts, feature lists, and hyperparameters are provided in Supplementary Materials.

\paragraph{Unified evaluation protocol for baselines.} All baseline methods share the same splits, seeds, early stopping, and preprocessing. CP/Tucker/PARAFAC2 ranks are selected by cross-validated BIC within an expanded grid $r\in\{3,5,7,9,11,13,15,17,19,21\}$ (with CORCONDIA sanity checks); NTF uses a temporal smoothness weight $\lambda_t\in\{0,10^{-3},10^{-2},10^{-1},1\}$; spectral/biclustering methods scan cluster counts $k\in\{6,8,10,12,14,16\}$ with identical regularization. We additionally include Block-Term Decomposition (BTD; block rank $\le 3$ per mode with ALS) and Tensor Spectral Clustering (TSC; normalized Laplacian on mode unfoldings with $k$ grid matched to spectral baselines). For fairness, we equalize total train-time budgets across methods and report wall-clock times on identical hardware. All methods use the same 10-fold CV and patience~$=10$ early stopping; random seeds are shared across folds; initialization uses 5 restarts with best validation score kept. Full hyperparameter search spaces for each baseline and \emph{wall-clock runtimes} per method are reported in Supplementary Tables~S1--S2.

\section{Background and Related Work}

Our work is situated at the intersection of three domains: the computational analysis of C. elegans morphogenesis \citep{sulston1983EmbryonicCellLineage,bao2006AutomatedCellLineage,murray2008AutomatedAnalysisEmbryonic,chisholm2005epidermal,kaletta2006FindingFunctionNovel}, co-clustering algorithms \citep{hartigan1972DirectClusteringData,madeira2004BiclusteringAlgorithmsBiological,xie2019ItTimeApplya,hochreiter2010FABIAFactorAnalysis,orzechowski2018EBICEvolutionarybasedParallel,yi2021COBRACFastImplementation}, and tensor-based data mining in biology \citep{kolda2009TensorDecompositionsApplications,cichocki2015TensorDecompositionsSignal,sidiropoulos2017TensorDecompositionSignal,drakopoulos2019TensorClusteringReview,sun2008IncrementalTensorAnalysis,cheng2019TensorBasedLowDimensionalRepresentation}.

Traditional approaches to analyzing developmental data rely on matricization, which may obscure multi-mode correlations present in tensor data \citep{kolda2009TensorDecompositionsApplications,cichocki2015TensorDecompositionsSignal}. Recent advances in tensor decomposition methods (CP/PARAFAC, Tucker, PARAFAC2) and tensor clustering approaches provide better frameworks for preserving inherent multi-dimensional structure \citep{sidiropoulos2017TensorDecompositionSignal,drakopoulos2019TensorClusteringReview}. In the context of C. elegans morphogenesis, previous studies have identified key developmental processes through manual annotation or single-mode analysis \citep{sulston1983EmbryonicCellLineage,bao2006AutomatedCellLineage,murray2008AutomatedAnalysisEmbryonic,zhu2024TIAM1RegulatesPolarized}, but comprehensive tensor-based approaches for discovering coordinated spatiotemporal patterns remain limited.

% ------------------------ RESULTS ------------------------
\section{Biological Validation on \textit{C.~elegans} Embryogenesis}
\label{sec:celegans_validation}

\subsection{Validation design and pre-registered behavioral criteria}
Note on label-free discovery: All discovery (clustering and merging) uses unlabeled data. Hyperparameters are tuned on discovery embryos using unsupervised criteria (stability and reconstruction) with multi-scale windows pre-registered prior to analysis; no biological labels or evaluation endpoints are used during tuning. Literature windows are used solely for preregistered evaluation on held-out embryos, avoiding any label leakage into the unsupervised learning process. The primary endpoints and thresholds (e.g., alignment window [225,235] min and $\theta=0.8$) were preregistered on OSF (anonymized link in Supplementary Materials; timestamped prior to analysis) based on literature ranges and simulation-based power checks.

Given the standardized tensor $\mathcal{X}\in\mathbb{R}^{C\times T\times F}$ (cells $\times$ time $\times$ features), the algorithm produces a collection of tri-clusters $\{\mathcal{B}_u\}_u$, each with index sets $(\mathcal{I}_u,\mathcal{J}_u,\mathcal{K}_u)$ and a blockwise low-rank approximation. We run $T_p$ randomized partitions/merges (Sec.~\ref{sec:methods}) and define a \emph{time-resolved co-clustering probability} between cells $i,j$,
\begin{equation}
P_{ij}(t)=\frac{1}{T_p}\sum_{\tau=1}^{T_p}\mathbf{1}\!\left[\exists\,u:\; i,j\in\mathcal{I}_u^{(\tau)}\ \wedge\ t\in\mathcal{J}_u^{(\tau)}\right],
\label{eq:coclust_prob}
\end{equation}
which is visualized as heatmaps in Figs.~\ref{fig:di_left}--\ref{fig:decline}. Feature importance within each tri-cluster is quantified by its normalized contribution to blockwise variance explained,
\begin{equation}
w_f = \frac{\sum_{(i,t)\in\mathcal{I}_u\times\mathcal{J}_u} X_{itf}^2}{\sum_{(i,t)\in\mathcal{I}_u\times\mathcal{J}_u}\sum_{k\in\mathcal{K}_u} X_{itk}^2},\quad f\in\mathcal{K}_u,
\label{eq:feature_weight}
\end{equation}
aggregated across blocks to produce Fig.~\ref{fig:features}.

\begin{figure}[t]
  \centering
  \includegraphics[width=\linewidth]{Demo1A_Dorsal_Left_Coclustering_Heatmap.png}
  \caption{\textbf{Dorsal Intercalation Left Side Co-clustering Heatmap.} Time-resolved co-clustering probability matrix for left-side dorsal intercalating cells during \textit{C.~elegans} embryonic morphogenesis (220--255 minutes relative to end of 4-cell stage). The heatmap displays pairwise clustering probabilities between 10 left-side hypodermal cells as they undergo convergent extension across the dorsal midline. High values (red) indicate synchronized clustering during 225--235 minutes; low values (blue) indicate independent movement. Cell identities correspond to hyp1L--hyp7L lineages.}
  \label{fig:di_left}
\end{figure}

\begin{figure}[t]
  \centering
  \includegraphics[width=\linewidth]{Demo1B_Dorsal_Right_Coclustering_Heatmap.png}
  \caption{\textbf{Dorsal Intercalation Right Side Co-clustering Heatmap.} Companion heatmap to Fig.~\ref{fig:di_left} for 10 right-side hypodermal cells (hyp1R--hyp7R). Peak co-clustering occurs during the window of basolateral protrusion extension and interdigitation across the midline.}
  \label{fig:di_right}
\end{figure}



\begin{figure}[t]
  \centering
  \includegraphics[width=\linewidth]{Demo4_Intestinal_Coclustering_Heatmap.png}
  \caption{\textbf{Temporal Decline Co-clustering Heatmap.} Co-clustering probability matrix for 12 dorsal cells (C01--C12) showing uniform high coordination during 225--240 minutes (red), followed by systematic decline during 240--255 minutes (blue), consistent with post-intercalation stabilization.}
  \label{fig:decline}
\end{figure}

\begin{figure}[t]
  \centering
  \includegraphics[width=\linewidth]{Demo6_Intestinal_Velocity_Field.png}
  \caption{\textbf{Intestinal Morphogenesis Velocity Field Analysis.} Velocity field of intestinal primordium cells during reorganization. Predominant negative $v_z$ velocities (inward/ventral movement toward imaging objective) reflect movements associated with apical constriction and internalization during E-lineage morphogenesis. Coordinate system: $x$ = anterior-posterior (AP), $y$ = left-right (LR), $z$ = dorsal-ventral (DV, imaging axis). Dynamics align with left--right asymmetry establishment and homotypic intercalation.}
  \label{fig:int_velocity}
\end{figure}

\begin{figure*}[t]
  \centering
  \includegraphics[width=.48\linewidth]{Demo7A_Dorsal_Coclustering_Features_Pie.png}\hfill
  \includegraphics[width=.48\linewidth]{Demo7B_Intestinal_Coclustering_Features_Pie.png}
  \caption{\textbf{Morphogenetic Feature Distributions.} (\textbf{A}) Dorsal intercalation: relative weights indicate Y-velocity (26\%), cell elongation (21\%), surface curvature (18\%), and additional geometric features. (\textbf{B}) Intestinal morphogenesis: Z-velocity (28\%), apical surface area (24\%), volume change (19\%), and ancillary parameters. Weights derived from blockwise variance contributions (Eq.~\ref{eq:feature_weight}).}
  \label{fig:features}
\end{figure*}

\begin{figure}[t]
  \centering
  \includegraphics[width=\linewidth]{Demo2_Dorsal_Cell_Trajectories.png}
  \caption{\textbf{Dorsal Intercalating Cell Trajectories.} Spatial trajectories showing convergent extension: left (blue) and right (red) cohorts migrate toward the dorsal midline, exhibiting characteristic finger-like interdigitation. Axes denote anterior--posterior (x) and left--right (y).}
  \label{fig:di_tracks}
\end{figure}

\paragraph{Primary endpoints (pre-specified).}
\begin{enumerate}
\item \textbf{Dorsal intercalation window detection.} A significant rise in $P_{ij}(t)$ among dorsal hyp cells on each side (left/right) during $t\in[225,235]$~min and sustained high values through $t\approx 254$~min (Figs.~\ref{fig:di_left},\ref{fig:di_right}), with cross-midline convergence demonstrated by directed trajectories (Fig.~\ref{fig:di_tracks}).
\item \textbf{Temporal coordination decay.} A systematic decline in $P_{ij}(t)$ after closure ($\gtrsim 240$~min), quantified by a negative slope $\beta<0$ from a robust regression of $\bar{P}(t)=\mathrm{mean}_{i\neq j}P_{ij}(t)$ (Fig.~\ref{fig:decline}).
\item \textbf{E-lineage internalization kinematics.} A left-shifted (negative) distribution of $v_z$ during Ea/Ep internalization and primordium compaction, with time-binned velocity fields showing dominant negative $z$-components (Fig.~\ref{fig:int_velocity}).
\end{enumerate}

\paragraph{Secondary endpoints.}
\begin{itemize}
\item \textbf{Left--right symmetry for hyp cells.} Similar $P_{ij}(t)$ envelopes for left vs.~right cohorts; interdigitation paths approaching the dorsal midline from opposite sides (Fig.~\ref{fig:di_tracks}).
\item \textbf{Feature attribution consistency.} Elevated weights for medial (Y) velocity, elongation, and curvature during dorsal intercalation; elevated weights for apical area, $v_z$, and volume change during intestinal internalization (Fig.~\ref{fig:features}).
\end{itemize}

\subsection{Recovering dorsal intercalation}
\label{subsec:val_dorsal}
Applying our tensor co-clustering to hyp-lineage cells and the window $t\in[220,255]$~min, we obtain side-specific tri-clusters whose cell--time blocks exhibit the characteristic synchronized rise of $P_{ij}(t)$ within $[225,235]$~min and maintenance through interdigitation (Figs.~\ref{fig:di_left},\ref{fig:di_right}). To quantify the window match, we define an \emph{alignment score}
\begin{equation}
\mathrm{Align}=\frac{1}{|\mathcal{W}|}\sum_{t\in\mathcal{W}}\mathbf{1}\!\left[\bar{P}(t)\geq \theta\right],\quad \mathcal{W}=[225,235]\ \mathrm{min},\ \theta=0.8,
\label{eq:align}
\end{equation}
yielding $\mathrm{Align}\approx 0.9$ for both left and right cohorts. A permutation test (time-label shuffling within embryos; $10^4$ permutations) shows the observed $\mathrm{Align}$ lies above the 99.9th percentile of the null.

Spatial trajectories (Fig.~\ref{fig:di_tracks}) show bilateral cohorts migrating toward $y=0$ with a consistent increase in medial speed $v_y$ during the high $P_{ij}(t)$ epoch, followed by positional stabilization after $t\approx 240$~min. Directional persistence and the left/right alternation of end positions match the expected "finger-like interdigitation".

\paragraph{Temporal coordination decay.}
We regress $\bar{P}(t)$ on time using Theil–Sen (robust) within $[240,255]$~min and obtain $\hat{\beta}<0$ with a 95\% CI excluding zero (Fig.~\ref{fig:decline}). Blockwise circular time-shifts (null) yield $p<10^{-3}$ after FDR correction (BY). This supports the literature-reported transition from coordinated interdigitation to post-closure stabilization.

\subsection{Recovering intestinal morphogenesis}
\label{subsec:val_int}
For E-lineage cells and $t\in[350,400]$~min, tri-clusters capture internalization and primordium organization: velocity fields (Fig.~\ref{fig:int_velocity}) display a dominant negative $v_z$ during $355$--$365$~min (Ea/Ep ingress), followed by moderated motions through $E16\to E20$. We test a one-sided bias in $v_z$ using time-binned medians (5-min bins) and confirm a significant negative shift during the internalization phase (bootstrap CI excludes zero; $p<10^{-4}$), consistent with apical constriction-driven inward movement. Subsequent high co-clustering probabilities within E-rings (not shown) coincide with homotypic intercalation and epithelialization of the intestinal tube.

\subsection{Feature-level concordance with mechanism}
\label{subsec:val_features}
We aggregate $w_f$ (Eq.~\ref{eq:feature_weight}) across tri-clusters contributing to each process. During dorsal intercalation, Y-axis velocity, cell elongation, and surface curvature account for the largest shares (Fig.~\ref{fig:features}A), matching a mechanics where medial convergence and wedge-like deformation dominate. During intestinal morphogenesis, negative $v_z$, apical surface area, and volume change are most informative (Fig.~\ref{fig:features}B), consistent with apical constriction, internalization, and apical domain remodeling. These attributions are reproduced when recomputing weights with leave-one-embryo-out and remain stable under 10–20\% missingness augmentation.

\subsection{Baseline method comparisons}

\subsubsection{Baseline methods}
\begin{itemize}
\item \textbf{Tensor decomposition baselines:} CP/PARAFAC with automatic rank selection, Tucker decomposition (HOSVD), PARAFAC2 for time-axis misalignment tolerance, and Non-negative Tensor Factorization (NTF) with temporal smoothness regularization.
\item \textbf{Matrix co-clustering baselines:} Spectral co-clustering (Dhillon et al.), Cheng-Church biclustering, PLAID overlapping biclustering, and time-continuity regularized biclustering.
\item \textbf{Sequential clustering baselines:} Time-slice clustering followed by temporal merging, and Dynamic Mode Decomposition (DMD) with clustering of temporal modes.
\end{itemize}
All baselines follow the unified evaluation protocol described in Methods.

\subsubsection{Evaluation metrics}
For each method, we measure:
\begin{itemize}
\item \textbf{Biological alignment:} Overlap with literature time windows for dorsal intercalation (225-235 min) and intestinal morphogenesis (350-400 min), measured by temporal Jaccard similarity and alignment score (Eq.~\ref{eq:align}).
\item \textbf{Trajectory consistency:} Correlation between predicted and observed velocity patterns ($v_y$ for dorsal intercalation, $v_z$ for intestinal morphogenesis).
\item \textbf{Feature attribution stability:} Rank correlation of feature importance scores across 10-fold cross-validation.
\item \textbf{Compression efficiency:} Explained variance ratio vs. number of components/blocks.
\end{itemize}

\subsubsection{Comparative results}
\textbf{Temporal alignment performance:} DiMergeTCC achieves significantly higher alignment scores compared to all baselines (Table~\ref{tab:baseline_comparison}). For dorsal intercalation, mean alignment score is 0.87 $\pm$ 0.05 vs. 0.62 $\pm$ 0.08 (Tucker), 0.58 $\pm$ 0.12 (CP), and 0.51 $\pm$ 0.15 (Dhillon spectral); effect sizes (Cohen's $d$) range from 2.1-3.8 (all $p < 0.001$, BY-adjusted permutation tests). For intestinal morphogenesis, DiMergeTCC achieves 0.83 $\pm$ 0.06 vs. 0.59 $\pm$ 0.11 (best baseline: PARAFAC2). We additionally evaluated Block-Term Decomposition and Tensor Spectral Clustering; results were consistent with the above trends and did not close the performance gap (see Supplementary Tables~S1--S2 for detailed metrics and wall-clock times).

\textbf{Velocity pattern recovery:} DiMergeTCC shows superior correlation with expected velocity patterns: $r = 0.89$ $\pm$ 0.04 for dorsal $v_y$ patterns vs. 0.67 $\pm$ 0.12 (Cheng-Church) and 0.61 $\pm$ 0.18 (Tucker). For intestinal $v_z$ patterns, DiMergeTCC achieves $r = 0.86$ $\pm$ 0.05 vs. 0.64 $\pm$ 0.13 (PARAFAC2).

\textbf{Feature attribution consistency:} Rank correlation of feature weights across cross-validation folds: DiMergeTCC = 0.91 $\pm$ 0.03, Tucker = 0.74 $\pm$ 0.08, CP = 0.69 $\pm$ 0.12, matrix methods = 0.52-0.67 range.

\textbf{Statistical significance:} All pairwise comparisons between DiMergeTCC and baselines achieve $p < 0.001$ (two-sided permutation tests, 10,000 iterations) with BY control across methods and metrics. Effect sizes are large (Cohen's $d > 0.8$) for all metrics, indicating practical significance beyond statistical significance.

\begin{table*}[t]
\centering
\caption{Baseline method comparison results (expanded baselines in Supplement).}
\label{tab:baseline_comparison}
\begin{tabular}{lcccc}
\toprule
\textbf{Method} & \textbf{Dorsal Align} & \textbf{Intestinal Align} & \textbf{Velocity Corr} & \textbf{Feature Stab} \\
\midrule
DiMergeTCC & 0.87 $\pm$ 0.05 & 0.83 $\pm$ 0.06 & 0.87 $\pm$ 0.04 & 0.91 $\pm$ 0.03 \\
Tucker/HOSVD & 0.62 $\pm$ 0.08** & 0.61 $\pm$ 0.09** & 0.69 $\pm$ 0.11** & 0.74 $\pm$ 0.08** \\
CP/PARAFAC & 0.58 $\pm$ 0.12** & 0.57 $\pm$ 0.14** & 0.65 $\pm$ 0.15** & 0.69 $\pm$ 0.12** \\
PARAFAC2 & 0.64 $\pm$ 0.09** & 0.59 $\pm$ 0.11** & 0.71 $\pm$ 0.09** & 0.76 $\pm$ 0.07** \\
NTF & 0.55 $\pm$ 0.11** & 0.54 $\pm$ 0.12** & 0.62 $\pm$ 0.13** & 0.67 $\pm$ 0.09** \\
Spectral (Dhillon) & 0.51 $\pm$ 0.15** & 0.48 $\pm$ 0.16** & 0.58 $\pm$ 0.17** & 0.61 $\pm$ 0.11** \\
Cheng-Church & 0.49 $\pm$ 0.14** & 0.52 $\pm$ 0.13** & 0.67 $\pm$ 0.12** & 0.59 $\pm$ 0.10** \\
PLAID & 0.46 $\pm$ 0.16** & 0.49 $\pm$ 0.15** & 0.61 $\pm$ 0.16** & 0.57 $\pm$ 0.12** \\
\bottomrule
\end{tabular}
\begin{tablenotes}
Values are mean $\pm$ standard deviation across 10-fold cross-validation. 
** BY-adjusted $p < 0.001$ vs. DiMergeTCC (two-sided permutation tests; family-wise adjustment across methods and metrics). BTD and TSC results are provided in Supplementary Tables~S1--S2 together with wall-clock times and parameter grids.
\end{tablenotes}
\end{table*}

\subsection{Comprehensive ablation studies and robustness analysis}
\label{subsec:val_robust}

\subsubsection{Algorithmic component ablations}
\begin{enumerate}
\item \textbf{No feature mode (matrix co-clustering):} Collapsing the feature dimension reduces dorsal alignment score from 0.87 to 0.65 (95\% CI: [0.58, 0.72], $p < 0.001$) and weakens $v_z$ bias detection (Cohen's $d$ drops from 1.8 to 1.1), confirming that the tensor structure is essential for separating kinematic vs. shape-driven regimes.

\item \textbf{No hierarchical merging:} Using only local tri-clustering without merging produces fragmented $P_{ij}(t)$ patterns and over-segmentation (mean cluster count: 47 vs. 12 with merging). Alignment scores drop by 28\% (95\% CI: [22\%, 35\%]), demonstrating the importance of global integration.

\item \textbf{Stricter merge rule (ours) vs. product-only Jaccard):} Requiring per-mode thresholds $(\alpha_I,\alpha_J,\alpha_K)$ \emph{and} non-decreasing composite score within a bootstrap CI reduces over-merging and improves alignment by 6--9\% compared to using only the product $J_3$ (all BY-adjusted $p<0.01$).

\item \textbf{No probabilistic candidate generation:} Deterministic grid-based partitioning reduces alignment scores by 18-23\% across both morphogenetic processes ($p < 0.001$), validating the theoretical motivation for preservation probability guarantees.
\end{enumerate}

\subsubsection{Multi-scale temporal window sensitivity}
We systematically vary temporal window sizes beyond the default \{16, 32, 64\} frames:
\begin{itemize}
\item \textbf{Ultra-short windows (8 frames):} Capture micro-coordination events but increase noise; alignment scores drop 15\% due to temporal fragmentation.
\item \textbf{Extended windows (128 frames):} Provide temporal stability but miss short-duration events; alignment scores decrease 12\% as fine-grained dynamics are averaged out.
\item \textbf{Optimal range:} 16--64 frame windows show peak performance, with 32-frame windows optimal for dorsal intercalation (duration $\sim$30 min) and 64-frame windows optimal for intestinal morphogenesis (duration $\sim$50 min). The efficiency curve in Supplementary Fig.~S3 shows a broad plateau across 24--64 frames, with a shallow optimum around 32--48 frames and degradation for both shorter and longer windows due to fragmentation and over-smoothing, respectively.
\end{itemize}

\subsubsection{Preprocessing and normalization sensitivity}
\begin{enumerate}
\item \textbf{Standardization strategy comparison:}
\begin{itemize}
    \item Global z-score (current): Alignment = 0.87 $\pm$ 0.05
    \item Per-embryo z-score: Alignment = 0.82 $\pm$ 0.07 (5.7\% decrease, $p = 0.02$)
    \item Robust scaling (median/IQR): Alignment = 0.85 $\pm$ 0.06 (2.3\% decrease, $p = 0.18$)
    \item Min-max scaling: Alignment = 0.79 $\pm$ 0.08 (9.2\% decrease, $p < 0.001$)
\end{itemize}

\item \textbf{Missing data interpolation methods:}
\begin{itemize}
    \item Nearest-neighbor (current): $v_z$ effect size = 1.8 $\pm$ 0.2
    \item Linear interpolation: $v_z$ effect size = 1.6 $\pm$ 0.3 (11\% decrease)
    \item Kalman filtering: $v_z$ effect size = 1.7 $\pm$ 0.2 (6\% decrease)
    \item Gaussian Process: $v_z$ effect size = 1.9 $\pm$ 0.2 (6\% improvement, not significant)
\end{itemize}
\end{enumerate}

\subsubsection{Feature group masking experiments}
We systematically mask entire feature categories to assess their contribution:
\begin{table*}[h]
\centering
\caption{Feature group ablation results ($\Delta$ alignment score).}
\label{tab:feature_ablation}
\begin{tabular}{lcc}
\toprule
\textbf{Masked Feature Group} & \textbf{Dorsal Intercalation} & \textbf{Intestinal Morphogenesis} \\
\midrule
None (full model) & 0.87 $\pm$ 0.05 & 0.83 $\pm$ 0.06 \\
Morphological features & -0.15 $\pm$ 0.04** & -0.21 $\pm$ 0.05** \\
3DCSQ shape features & -0.12 $\pm$ 0.03** & -0.08 $\pm$ 0.03* \\
Velocity/kinematic features & -0.22 $\pm$ 0.04** & -0.35 $\pm$ 0.06** \\
Contact/topology features & -0.06 $\pm$ 0.02* & -0.04 $\pm$ 0.02 \\
\bottomrule
\end{tabular}
\begin{tablenotes}
* $p < 0.05$, ** $p < 0.001$ vs. full model (BY-adjusted).
\end{tablenotes}
\end{table*}

\textbf{Key insights:} Kinematic features show the largest impact for both processes, confirming that velocity patterns are primary discriminators. Morphological features are more critical for intestinal morphogenesis (volume/shape changes), while 3DCSQ features contribute more to dorsal intercalation (surface deformation patterns).

\subsubsection{Synthetic validation and power analysis}
To validate detection sensitivity, we inject synthetic tri-blocks with known ground truth:
\begin{itemize}
\item \textbf{Block size sensitivity:} Minimum detectable block size is 8$\times$12$\times$15 (cells$\times$time$\times$features) at 90\% power with $\alpha = 0.05$.
\item \textbf{Signal-to-noise ratio:} Detection power drops from 95\% to 50\% as SNR decreases from 2.0 to 1.2, establishing practical operating ranges.
\item \textbf{Temporal fragmentation:} Blocks split across 2-3 temporal segments are recovered with 85\% accuracy; >3 segments reduce accuracy to 65\%.
\end{itemize}

\subsubsection{Cross-embryo stability and external validation}
\begin{itemize}
\item \textbf{Leave-one-embryo-out validation:} Mean tri-modal Jaccard similarity = 0.68 $\pm$ 0.12 across N=8 embryos (95\% CI: [0.61, 0.75]).
\item \textbf{Batch effect assessment:} Imaging sessions separated by >1 week show 8-12\% reduced alignment scores, but core patterns remain significant ($p < 0.001$).
\item \textbf{Synthetic missingness tolerance:} Alignment scores remain >0.80 up to 20\% additional missing data; degradation becomes significant at 25\% ($p < 0.01$).
\end{itemize}

\subsubsection{Statistical null distributions and multiple testing}
\begin{itemize}
\item \textbf{Permutation strategy:} 10,000 circular time shifts per embryo preserve temporal autocorrelation structure while breaking biological coordination; we also apply per-cell independent shifts and Fourier phase randomization for conservative nulls when needed.
\item \textbf{FDR control:} Benjamini--Yekutieli procedure controls false discovery rate at $\leq 5\%$ across 180 statistical tests (45 tri-clusters $\times$ 4 validation metrics). We report BY-adjusted $q$-values; when positive dependence is verified (via eigenvalue decay and cross-block HSIC tests), we additionally report BH $q$-values as a sensitivity analysis, while confirmatory claims rely strictly on BY-adjusted $p$-values for valid control under arbitrary dependence.
\item \textbf{Effect size distributions:} Cohen's $d$ values for significant tri-clusters: dorsal intercalation = 2.1 $\pm$ 0.4, intestinal morphogenesis = 1.8 $\pm$ 0.3 (both large effect sizes).
\end{itemize}

All ablations demonstrate that DiMergeTCC's performance depends on the integrated combination of tensor structure, probabilistic candidate generation, hierarchical merging with per-mode safeguards, and appropriate preprocessing. No single component dominates, but kinematic features and temporal window selection are most critical for biological pattern recovery.

\subsection{Conclusion}
Our tensor co-clustering framework recovers the hallmark spatiotemporal coordination of \emph{dorsal intercalation} and the kinematic/shape signatures of \emph{intestinal morphogenesis}, with quantitative alignment to the known developmental schedule and mechanisms. The agreement spans probability dynamics, trajectories, velocity fields, and feature attributions, and is robust to ablations and null tests. These results provide strong evidence that the method captures genuine biological programs rather than artifacts of partitioning or feature selection.


% -------------------- Figure environments (captions as provided) --------------------



% ------------------------ FUTURE WORK ------------------------
\section{Future Work: Careful Discovery of New Spatiotemporal Patterns}
\label{sec:future}

We briefly outline immediate next steps while deferring details to the Supplementary Materials: (i) unsupervised discovery under strict regularization, (ii) pre-registered testing on held-out embryos or perturbations, and (iii) mechanistic triangulation by orthogonal readouts (genetics, force/signaling reporters). Methodological extensions include uncertainty-aware co-clustering, physics/topology-informed features, improved alignment across embryos/species, streaming/federated scaling, rare-event scan statistics, and causal triangulation. Validation standards will include positive/negative controls, ablations, calibrated power analyses, and full uncertainty reports. Full protocols and ablation menus are provided in the Supplement.

% ------------------------ DISCUSSION ------------------------
\section{Discussion}\label{sec:discussion}
DiMergeTCC discovers contiguous, multi-way coordination without prior labels, bridging matrix co-clustering and tensor decompositions. Blocks align with well-documented dorsal intercalation and E-lineage morphogenesis, while revealing post-event coordination decay. We emphasize that our theoretical analysis provides calibrated upper bounds under explicit assumptions, not absolute guarantees. Extensions include higher-order tensors (adding embryos/perturbations), block-overlap priors with per-mode safeguards, and causal tests across genotypes.

% ------------------------ LIMITATIONS ------------------------
\section{Limitations}\label{sec:limits}
Current scoring assumes sub-Gaussian noise and relies on time-contiguity (or a small union of contiguous windows as noted in Sec.~\ref{sec:methods}). Extremely sparse or highly irregular sampling may reduce power; future work will incorporate continuous-time models and robust losses. Results may also be influenced by feature engineering; we therefore provide feature-subset ablations and random-subset robustness analyses (Supplementary Table~S3) and axis-jitter sensitivity checks (Supplementary Fig.~S1b).

% ------------------------ BIBLIO ------------------------
\bibliographystyle{abbrvnat}
\bibliography{refs}

\end{document}